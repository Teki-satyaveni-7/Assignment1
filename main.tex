\let\negmedspace\undefined
\let\negthickspace\undefined
\documentclass{article}
\usepackage{cite}
\usepackage{amsmath,amssymb,amsfonts,amsthm}
\usepackage{algorithmic}
\usepackage{graphicx}
\usepackage{textcomp}
\usepackage{xcolor}
\usepackage{txfonts}
\usepackage{float}
\usepackage{listings}
\usepackage{enumitem}
\usepackage{mathtools}
\usepackage{gensymb}
\usepackage{tfrupee}
\usepackage[breaklinks=true]{hyperref}
\usepackage{tkz-euclide} % loads  TikZ and tkz-base
\usepackage{listings}
\usepackage{gvv}
%
%\usepackage{setspace}
%\usepackage{gensymb}
%\doublespacing
%\singlespacing

%\usepackage{graphicx}
%\usepackage{amssymb}
%\usepackage{relsize}
%\usepackage[cmex10]{amsmath}
%\usepackage{amsthm}
%\interdisplaylinepenalty=2500
%\savesymbol{iint}
%\usepackage{txfonts}
%\restoresymbol{TXF}{iint}
%\usepackage{wasysym}
%\usepackage{amsthm}
%\usepackage{iithtlc}
%\usepackage{mathrsfs}
%\usepackage{txfonts}
%\usepackage{stfloats}
%\usepackage{bm}
%\usepackage{cite}
%\usepackage{cases}
%\usepackage{subfig}
%\usepackage{xtab}
%\usepackage{longtable}
%\usepackage{multirow}
%\usepackage{algorithm}
%\usepackage{algpseudocode}
%\usepackage{enumitem}
%\usepackage{mathtools}
%\usepackage{tikz}
%\usepackage{circuitikz}
%\usepackage{verbatim}
%\usepackage{tfrupee}
%\usepackage{stmaryrd}
%\usetkzobj{all}
%    \usepackage{color}                                            %%
%    \usepackage{array}                                            %%
%    \usepackage{longtable}                                        %%
%    \usepackage{calc}                                             %%
%    \usepackage{multirow}                                         %%
%    \usepackage{hhline}                                           %%
%    \usepackage{ifthen}                                           %%
  %optionally (for landscape tables embedded in another document): %%
%    \usepackage{lscape}
%\usepackage{multicol}
%\usepackage{chngcntr}
%\usepackage{enumerate}

%\usepackage{wasysym}
%\documentclass[conference]{IEEEtran}
%\IEEEoverridecommandlockouts
% The preceding line is only needed to identify funding in the first footnote. If that is unneeded, please comment it out.

\newtheorem{theorem}{Theorem}[section]
\newtheorem{problem}{Problem}
\newtheorem{proposition}{Proposition}[section]
\newtheorem{lemma}{Lemma}[section]
\newtheorem{corollary}[theorem]{Corollary}
\newtheorem{example}{Example}[section]
\newtheorem{definition}[problem]{Definition}
%\newtheorem{thm}{Theorem}[section]
%\newtheorem{defn}[thm]{Definition}
%\newtheorem{algorithm}{Algorithm}[section]
%\newtheorem{cor}{Corollary}
\newcommand{\BEQA}{\begin{eqnarray}}
\newcommand{\EEQA}{\end{eqnarray}}
%\newcommand{\define}{\stackrel{\triangle}{=}}
\theoremstyle{remark}
\newtheorem{rem}{Remark}

%\bibliographystyle{ieeetr}
\begin{document}
\title{LATEX ASSIGNMENT}
\author{TEKI SATYAVENI}
\date{26-09-2023}
\maketitle
\section*{CLASS 10} 
\date{}
\centering
\subsection*{PROBABILITY}


\begin{enumerate}[label=\arabic*.,ref=\theenumi]
\item Two dice are thrown simultaneously. The probability that the sum of two numbers appearing on the top of the dice is less than $12$, is
 \begin{enumerate}
        \item $\frac{1}{36}$
        \item $\frac{35}{36}$
        \item $0$
        \item $1$
    \end{enumerate}

\item A jar contains $18$ marbles. Some are red and others are yellow. If a 
marble is drawn at random from the jar, the probability that it is red is $\frac{2}{3}$. Find the number of yellow marbles in the jar.

\item A die is thrown twice. What is the probability that 
\begin{enumerate}[label=(\roman*)]
 \item $5$ will come up at least once, and 
 \item $5$ will not come up either time ? 
\end{enumerate}

\item Let $A$ and $B$ be two events such that $P(A)=\frac{5}{8}$, $P(B)=\frac{1}{2}$ and $P(A/B)=\frac{3}{4}$. Find the value of $P(B/A)$.

\item Two balls are drawn at random from a bag containing $2$ red balls and $3$ blue balls, without replacement. Let the variable $X$ denotes the number of red balls. Find the probability distribution of $X$.

\item A card from a pack of $52$ playing cards is lost. From the remaining cards, $2$ cards are drawn at random without replacement, and are found to be both aces. Find the probability that lost card being an ace.

\item Probabilities of $A$ and $B$ solving a specific problem are $\frac{2}{3}$ and $\frac{3}{5}$, respectively. If both of them try independently to solve the problem, then 
find the probability that the problem is solved.

\item A pair of dice is thrown. It is given that the sum of numbers appearing on both dice is an even number. Find the probability that the number appearing on at least one die is $3$.

\item In \figref{fig:fig1.png},At the start of a cricket match, a coin is tossed and the team winning the 
toss has the opportunity to choose to bat or bowl. Such a coin is unbiased 
with equal probabilities of getting head and tail.

\begin{figure}[H]
        \centering
        \includegraphics[width=\columnwidth]{./figs/Screenshot (19).png}
        \caption{Tossing a coin}
        \label{fig:fig1.png}
    \end{figure}

Based on the above information, answer the following questions :
\begin{enumerate}[label=(\alph*)]
 \item  If such a coin is tossed $2$ times, then find the probability 
distribution of number of tails.
 
 \item Find the probability of getting at least one head in three tosses of 
such a coin. 
\end{enumerate}

\item Two cards are drawn successively with replacement from a well shuffled pack of $52$ cards. Find the probability distribution of the number of spade cards.

\item A pair of dice is thrown and the sum of the numbers appearing on the dice is observed to be $7$. Find the probability that the number $5$ has appeared on atleast one die.

\item In \figref{fig:fig2.png}, A shopkeeper sells three types of flower seeds $A1$, $A2$, $A3$. They are sold in the form of a mixture, where the proportions of these seeds are $4:4:2$, respectively. The germination rates of the three types of seeds are $45\%$, $60\%$ and $35\%$ respectively.

\begin{figure}[H]
        \centering
        \includegraphics[width=\columnwidth]{./figs/Screenshot (23).png}
        \caption{Three Types of Flower Seeds}
        \label{fig:fig2.png}
    \end{figure}

    Based on the above information:
    
    \begin{enumerate}[label=(\alph*)]
    
 \item Calculate the probability that a randomly chosen seed will germinate;
 
 \item Calculate the probability that the seed is of type $A2$, given that a randomly chosen seed germinates.

\end{enumerate}

\item Three friends $A$, $B$ and $C$ got their photograph clicked. Find the 
probability that $B$ is standing at the central position, given that $A$ is 
standing at the left corner. 

\item In \figref{fig:fig3.png} A coin is tossed twice. The following table shows the probability 
distribution of number of tails :
\begin{figure}[H]
        \centering
        \includegraphics[width=\columnwidth]{./figs/Screenshot (28).png}
        \caption{Probability Distribution of number of tails}
        \label{fig:fig3.png}
    \end{figure}

    \begin{enumerate}[label=(\alph*)]
    
 \item  Find the value of $K$. 
 
 \item  Is the coin tossed biased or unbiased ? Justify your answer.

\end{enumerate}

\item In \figref{fig:fig4.png} In a game of Archery, each ring of the Archery target is valued. The 
centre most ring is worth $10$ points and rest of the rings are allotted 
points $9$ to $1$ in sequential order moving outwards.

Archer A is likely to earn $10$ points with a probability of $0·8$ and Archer $B$ 
is likely the earn $10$ points with a probability of $0·9$.

\begin{figure}[H]
        \centering
        \includegraphics[width=\columnwidth]{./figs/Screenshot (26).png}
        \caption{Ring of the Archery Target}
        \label{fig:fig4.png}
    \end{figure}

Based on the above information, answer the following questions : 
If both of them hit the Archery target, then find the probability that 

\begin{enumerate}[label=(\alph*)]
    
 \item  exactly one of them earns $10$ points.
 
 \item  both of them earn $10$ points.

\end{enumerate}


\item 
\begin{enumerate}[label=(\alph*)]
    
 \item  Events $A$ and $B$ are such that
 P(A) =  $\frac{1}{2}$ , P(B) =  $\frac{7}{12}$  and $ P( \overline{A}  \cup  \overline{B} )= \frac{1}{4}$ Find whether the events $A$ and $B$ are independent or not.
 
 \item  A box $B_{1}$ contains $1$ white ball and $3$ red balls.Another box $B_{2}$ contains $2$ white balls and $3$ red balls.If one ball is drawn at random from each of the boxes $B_{1}$ and $B_{2}$ then find the probability that the two balls drawn are of the same colour.
 
\end{enumerate}

 \item There are two boxes, namely box-I and box-II. Box-I contains $3$ red and $6$ black balls. Box-II contains $5$ red and $5$ black balls. One of the two boxes, is selected at random and a ball is drawn at random. The ball drawn is found to be red. Find the probability that this red ball comes out from box-II.

\item In a toss of three different coins, find the probability of coming up of three heads, if it is known that at least one head comes up.

\item Two rotten apples are mixed with $8$ fresh apples. Find the probability distribution of number of rotten apples, if two apples are drawn at random, one-by-one without replacement.

\item A laboratory blood test is $98\%$ effective in detecting a certain 
disease when it is in fact, present. However, the test also yields 
a false positive result for $0·4\%$ of the healthy person tested. 
From a large population, it is given that 0·2$\%$ of the population 
actually has the disease. 
Based on the above, answer the following questions : 

  \begin{enumerate}[label=(\alph*)]
    
 \item One person, from the population, is taken at random and 
given the test. Find the probability of his getting a 
positive test result.  
 
 \item  What is the probability that the person actually has the 
disease, given that his test result is positive ?

\end{enumerate}

\item Two cards are drawn from a well-shuffled pack of playing 
cards one-by-one with replacement. The probability that the 
first card is a king and the second card is a queen is 

\begin{enumerate}[label=(\alph*)]
    
 \item $\frac{1}{13} + \frac{1}{13}$
 
 \item $\frac{1}{13} \times \frac{4}{51}$

 \item $\frac{4}{52} \times \frac{3}{51}$
 
 \item $\frac{1}{13} \times \frac{1}{13}$ 

\end{enumerate}

\item In \figref{fig:fig5.png} If $X$ is a random variable with probability distribution as given 
below :
\begin{figure}[H]
        \centering
        \includegraphics[width=\columnwidth]{./figs/Screenshot (32).png}
        \caption{Probability Distribution}
        \label{fig:fig5.png}
    \end{figure}

The value of $k$ and the mean of the distribution respectively 
are

 \begin{enumerate}[label=(\alph*)]
    
 \item  $\frac{1}{7}$,1 
 
 \item  $\frac{1}{6}$,2

 \item  $\frac{1}{6}$,1

 \item  $\frac{1}{6}$

\end{enumerate}


\item For two events $A$ and $B$ if P(A) = $\frac{4}{10}$, P(B) = $\frac{8}{10}$ and 
$ P(B \mid A)$=$\frac{6}{10}$, then find $ P(A \cup B)$.

\item Bag I contains $4$ red and $3$ black balls. Bag II contains $3$ red 
and $5$ black balls. One of the two bags is selected at random 
and a ball is drawn from the bag, which is found to be red. 
Find the probability that the ball is drawn from Bag II.

\item Two cards are drawn successively without replacement from a 
well-shuffled pack of $52$ cards. Find the probability 
distribution of the number of aces and hence find its mean.
\newpage

\item The probability of solving a specific question independently by $A$ and $B$ 
are $\frac{1}{3}$ and $\frac{1}{5}$ respectively. If both try to solve the question independently, 
the probability that the question is solved is 

\begin{enumerate}[label=(\alph*)]
    
 \item  $\frac{7}{15}$
 
 \item  $\frac{8}{15}$
 
 \item  $\frac{2}{15}$
 
 \item  $\frac{14}{15}$

\end{enumerate}

\item A card is picked at random from a pack of $52$ playing cards. Given that 
the picked up card is a queen, the probability of it being a queen of 
spades is ?

\item A bag contains $19$ tickets, numbered $1$ to $19$. A ticket is drawn at random 
and then another ticket is drawn without replacing the first one in the 
bag. Find the probability distribution of the number of even numbers on 
the ticket.

\item Find the probability distribution of the number of successes in two tosses 
of a die, when a success is defined as ‘‘number greater than $5$’’.

\item The random variable $X$ has a probability function $P(x)$ as defined below, 
where $k$ is some number :

\begin{align}
    p(x) = \begin{cases}
        k, & \text{if } x = 0, \\
        2k, & \text{if } x = 1, \\
        3k, & \text{if } x = 2, \\
        0, & \text{otherwise.}
    \end{cases}
\end{align}

Find :
\begin{enumerate}[label=(\roman*)]
 \item The value of $k$
 
 \item $P(X < 2)$, $P(X \leq 2)$, $P(X\ \geq 2)$
 
 \end{enumerate}

\item Consider the following hypothesis :

\begin {align}
H0 : \mu =  35\\
H1 : \mu \neq 35
\end{align}
A sample of $81$ items is taken whose mean is $37·5$ and the standard deviation is $5$. Test the hypothesis at $5\%$ level of significance.

[Given : Critical value of $Z$ for a two-tailed test at $5\%$ level of significance is $1.96$]

\item In \figref{fig:fig6.png} Fit a straight line trend by the method of least squares and find the trend 
value for the year $2008$ for the following data :

\begin{figure}[H]
        \centering
        \includegraphics[width=\columnwidth]{./figs/Screenshot (37).png}
        \caption{Years and Production}
        \label{fig:fig6.png}
    \end{figure}
\end{enumerate}
\end{document}
